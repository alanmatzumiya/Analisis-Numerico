 \begin{document}	
 	\begin{enumerate}
 	
 	\item
 	Hacer un script de python que reprodusca la tabla 2.1 pg 22 del libro de texto.	\\
 	Example 2.1\\
 	Use a step size $h$ to develop an approximate solution to the IVP
 	   \begin{align*}
 	     x'(t) = &(1 - 2t)x(t) , t>0 \\
 	     x&(0) = 1,
 	   \end{align*}
 	over the interval $0 \leq t \leq 0.9$ . \\
 	We have purposely chosen a problem with a known exact solution \
 	   
 	   \begin{align*}
	 	  x(t) = exp( 14 - ( 12 - t)^2 ), 
	   \end{align*}
 	      
 	\item
 	Crear un script para Generar tabla 2.2 y figura 2.3 de la pg 25.\\
 	Example 2.2\\
 	Use Euler’s method to solve the IVP
 	    \begin{align*}
 	    x'(t) = 2x&(t)( 1 - x(t)) , t > 10, \\
 	    x&(10) = 15, 
 	    \end{align*}    
 	for $ 10 \leq t \leq 11 $ with $h = 0.2$ .
 		
 	\item 
 	Obtain the recurrence relation that enables $ x_{n+1}$ to be calculated
 	from $x_n$ when Euler’s method is applied to the IVP  $x'(t) = \lambda x(t)$,
 	$x(0) = 1$ with $\lambda = - 10 $ . In each of the cases $h = 1/6$ and $h = 1/12$
 	\item[(a)]
 	calculate $x_1$ , $x_2$ and $x_3$ , \\
 	\item[(b)]
 	Plot the points $(t_0 , x_0 )$, $(t_1 , x_1 )$, $(t_2 , x_2 )$, and $(t_3 , x_3 )$ and compare
 	with a sketch of the exact solution  $x(t) = e^{\lambda t}$ 
 	
 	Comment on your results. What is the largest value of h that can be used when $\lambda = - 10 $ to ensure that $x_n > 0$ for all $n = 1, 2, 3, . . .$?
 	
 	\item
 	This question concerns approximations to the IVP
 	  \begin{align*}
 	     x''(t) + 3&x'(t) + 2x(t) = t^{2} , t > 0 \\
 	     &x(0) = 1, x'(0) = 0
 	  \end{align*}
 	\item[(a)]
 	Description write the above initial value problem as a first-order system
 	and hence derive Euler’s method for computing approximations to $x(t_{n+1})$ 
 	and $x'(t_{n+1})$ in terms of approximations to $x(t_n)$ and $x'(t_n)$ .\\
 	Si definimos una nueva variable como $y(t) = x'(t)$ , entonces $y'(t) = x''(t)$.
 	De esta manera obtenemos el siguiente sistema de primer orden con sus respectivas condiciones
 	iniciales :
 	 \begin{align*}
 	 x'&(t) = y(t),   x(0) = 1 \\
 	 y'&(t) = t^2 - 3y(t) - 2x(t) = t^{2} , y(0) = 0 
 	 \end{align*}
 	 	Si consideramos las siguientes funciones\\
 	 	\begin{align*}
 	 	f&(x(t_n), y(t_n), t_n) = y(t_n) \\
 	 	g&(x(t_n), y(t_n), t_n) = (t_n)^2 - 3y(t_n) - 2x(t_n)
 	 	\end{align*}
 	 	El metodo de Euler para cada ecuacion seria el siguiente:
 	 	
 	 	\begin{align*}
 	 	x_{n+1} &= x_n + hf(x(t_n), y(t_n), t_n) \\
 	 	y_{n-1} &= y_n + hg(x(t_n), y(t_n), t_n)   
 	 	\end{align*}	
 	 	
 	
 	\item[(b)]
 	By eliminating y, show that the system
 	  \begin{align*}
 		x'(t) = y(t) - 2x(t) \\
 		y'(t) = t^2 - y(t) 
 	  \end{align*}
 	  
 	has the same solution x(t) as the IVP (2.18) provided that
 	$x(0) = 1$, and that $y(0)$ is suitably chosen. What is the appropriate value of $y(0)$?\\
 	
 	Para valores de $y(0)$ cercanos al cero se tiene buena aproximacion\\
 	
 	\item[(c)]
 	Apply Euler’s method to the system in part (b) and give formulate for computing 
 	approximations to $x(t_{n+1})$ and $y(t_{n+1})$ in terms of approximations to $x(t_n)$ and $y(t_n)$.
 		 	Si consideramos las siguientes funciones\\
 		 	\begin{align*}
 		 	f&(x(t_n), y(t_n), t_n) = y(t_n) - 2x(t_n) \\
 		 	g&(x(t_n), y(t_n), t_n) = (t_n)^2 - y(t_n) 
 		 	\end{align*}
 		 	El metodo de Euler para cada ecuacion seria el siguiente:
 		 	
 		 	\begin{align*}
 		 	x_{n+1} &= x_n + hf(x(t_n), y(t_n), t_n) \\
 		 	y_{n-1} &= y_n + hg(x(t_n), y(t_n), t_n)  
 		 	\end{align*}
 
 	 	\item[(d)] 
 	Show that the approximations to $x(t_2)$ produced by the methods
 	in (a) and (c) are identical provided both methods use the same
 	value of $h$.
 	\item
 	Prove that $e^x \geq 1 + x$ for all $x \geq 0$. [Hint: use the fact that $e^t \geq 1$
 	for all $t \geq 0$ and integrate both sides over the interval $0 \leq t \leq x$
 	(where $x \geq 0$.)]
 	\end{enumerate}
 	
	 	Utilizando la desigualdad $e^t \geq 1$ , se puede observar que las funciones de
	 	ambos lados son continuas en el intervalo $0 \leq t \leq x$ . Por lo tanto la desigualdad es integrable en este mismo intervalo es decir, 
 	
 	\[
 	\int_{0}^{x} e^{t} \, dx \geq \int_{0}^{x} 1 \, dx 
    \Longrightarrow e^x - 1 \geq x \\
 	\]
 	
 	Por lo tanto $e^x \geq x + 1$, Para toda $x\geq0$
 \end{document}
 
